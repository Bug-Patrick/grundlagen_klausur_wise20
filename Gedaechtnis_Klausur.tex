% !TeX program = lualatex

\documentclass[
	ngerman,
	color = black,
	]{tudaexercise}

\usepackage[english, main=ngerman]{babel}
\usepackage[autostyle]{csquotes}

\usepackage{biblatex}
\bibliography{DEMO-TUDaBibliography}

%Formatierungen für Beispiele in diesem Dokument. Im Allgemeinen nicht notwendig!
\let\file\texttt
\let\code\texttt
\let\pck\textsf
\let\cls\textsf
\let\tbs\textbackslash

\ConfigureHeadline{
	headline={Name:  Matrikelnummer: }
}

\begin{document}

\title[Übung TUDaExercise]{Klausur Grundlagen des CAE/CAD}
%\subtitle{Untertitel}
\author{Prof. Dr.-Ing. R. Anderl}
\term{Wintersemester 2021}
\date{17.04.2020}

\maketitle

\begin{task}[credit=13 P]{Geometriemodelle}
	\begin{subtask}[credit=4 P]
		Das abgebildete Bauteil wurde mit verschiedenen Features (Konstruktionselementen) modelliert.
		Benennen Sie die einzelnen Features und tragen Sie dies unter Berücksichtigung der Reihenfolge (vom Groben ins Feine) in die untenstehende Liste eindeutig ein, sodass eine Zuordnung der Features zur Geometrie möglich ist.
		
		\textit{Hinweis: Keine Features auf der Rückseite.} % zus. ?
	\end{subtask}
	
	\begin{subtask}[credit=9 P]
		Erstellen sie anhand der unten gegebenen Volumenprimitiven (Q und Z) den CSG-Baum für das abgebildete Bauteil.
		Verwenden Sie bekannte mengentheoretische Operatoren.
	\end{subtask}

	
\end{task}

\begin{task}[credit=21 P]{Geometrische Modellierung und Volumenmodelle}
	\begin{subtask}[credit=6 P]
		Nennen Sie Merkmale geometrischer Modellierung.
		Verwenden Sie das gelernte Ordnungsschema.
	\end{subtask}
	
	\begin{subtask}[credit=2{,}5 P]
		Ein wichtige Geometrieelement sind die Freiformkurven.
		Welche Freiformkurven wurden im Rahmen der Vorlesung behandelt und in welche 2 Kategorien lassen sich die Kurven unterscheiden.
	\end{subtask}
	
	\begin{subtask}[credit=1{,}5 P]
		Welche Möglichkeiten zur Beeinflussung der Kurv gibt Hermitkurve.
		Skizzieren sie ein Beispiel.
	\end{subtask}
	
	\begin{subtask}[credit=6 P]
		
	\end{subtask}
	
	\begin{subtask}[credit=5 P]
		
	\end{subtask}
	
\end{task}

\begin{task}[credit=9 P]{Geometrische Modellierung}
	\begin{subtask}[credit=2 P]
		geometrische Transformation im 3d Raum; Wieso 4*4 und ...
	\end{subtask}
	
	\begin{subtask}[credit=2 P]
		Matrizen für Translation und Rotation um y-Achse aufstellen
	\end{subtask}
		
	\begin{subtask}[credit=3 P]
		Punkt (1,7,4) translieren um $d_x=1$, $d_y=3$ und $d_z=4$
	\end{subtask}
	
	\begin{subtask}[credit=2 P]
		Ergebnis von voriger Aufgabe 30° um die z-Achse drehen
	\end{subtask}
\end{task}

\begin{task}[credit=15{,}5 P]{Datenmanagementsysteme}
	\begin{subtask}[credit=5 P]
		
	\end{subtask}
	
	\begin{subtask}[credit=2 P]
		
	\end{subtask}
	
	\begin{subtask}[credit=2{,}5 P]
		
	\end{subtask}
	
	\begin{subtask}[credit=4 P]
		
	\end{subtask}
	
	\begin{subtask}[credit=2 P]
		
	\end{subtask}
\end{task}

\begin{task}[credit=18{,}5 P]{CAE-Prozessketten}
	\begin{subtask}[credit=2 P]
		
	\end{subtask}
	
	\begin{subtask}[credit=4 P]
		
	\end{subtask}	
	
	\begin{subtask}[credit=4 P]
		
	\end{subtask}
	
	\begin{subtask}[credit=2{,}5 P]
		
	\end{subtask}
	
	\begin{subtask}[credit=4 P]
		
	\end{subtask}
	
	\begin{subtask}[credit=2 P]
		
	\end{subtask}
\end{task}

\begin{task}[credit=23 P]{FEM}
	\begin{subtask}[credit=5 P]
		
	\end{subtask}
	
	\begin{subtask}[credit=2 P]
		
	\end{subtask}
	
	\begin{subtask}[credit=16 P]
		
	\end{subtask}
\end{task}

\end{document}
