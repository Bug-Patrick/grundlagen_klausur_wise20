% !TeX program = lualatex

\documentclass[
	ngerman,
	color = black,
	]{tudaexercise}

\usepackage[english, main=ngerman]{babel}
\usepackage[autostyle]{csquotes}
\usepackage{inputenc}

\ConfigureHeadline{
	headline={Name:  Matrikelnummer: }
}

\begin{document}

\title[Übung TUDaExercise]{Klausur Grundlagen des CAE/CAD}
%\subtitle{Untertitel}
\author{Prof. Dr.-Ing. R. Anderl}
\term{Wintersemester 2021}
\date{17.04.2020}

\maketitle

\begin{task}[credit=13 P]{Geometriemodelle}
	\begin{subtask}[credit=4 P]
		Das abgebildete Bauteil wurde mit verschiedenen Features (Konstruktionselementen) modelliert.
		Benennen Sie die einzelnen Features und tragen Sie dies unter Berücksichtigung der Reihenfolge (vom Groben ins Feine) in die untenstehende Liste eindeutig ein, sodass eine Zuordnung der Features zur Geometrie möglich ist.
		
		\textit{Hinweis: Keine Features auf der Rückseite.} % zus. ?
	\end{subtask}
	
	\begin{subtask}[credit=9 P]
		Erstellen sie anhand der unten gegebenen Volumenprimitiven (Q und Z) den CSG-Baum für das abgebildete Bauteil.
		Verwenden Sie bekannte mengentheoretische Operatoren.
	\end{subtask}

	
\end{task}

\begin{task}[credit=21 P]{Geometrische Modellierung und Volumenmodelle}
	\begin{subtask}[credit=6 P]
		Nennen Sie Merkmale geometrischer Modellierung.
		Verwenden Sie das gelernte Ordnungsschema.
	\end{subtask}
	
	\begin{subtask}[credit=2{,}5 P]
		Ein wichtige Geometrieelement sind die Freiformkurven.
		Welche Freiformkurven wurden im Rahmen der Vorlesung behandelt und in welche 2 Kategorien lassen sich die Kurven unterscheiden.
	\end{subtask}
	
	\begin{subtask}[credit=1{,}5 P]
		Welche Möglichkeiten zur Beeinflussung der Kurv gibt Hermitkurve.
		Skizzieren sie ein Beispiel.
	\end{subtask}
	
	\begin{subtask}[credit=6 P]
		Volumenmodelle i L Volumina vollständig zu beschreiben und Körper zu definieren.
		Nennen Sie die verschiedenen Volumenmodelle basierend auf den unterschiedlichen Datenstrukturen.
		Beschreiben Sie jedes Volumenmodel in einem Satz.
	\end{subtask}
	
	\begin{subtask}[credit=5 P]
		Im Rahmen der Volumenmodelle ist der Unterschied zwischen Geometrie und Topologie wichtig.
		Wie sind Begr definiert?
		Stelle Sie jeweilige Elemente in einem Schaubild gegenüber.
	\end{subtask}
	
\end{task}

\begin{task}[credit=9 P]{Geometrische Modellierung}
	\begin{subtask}[credit=2 P]
		geometrische Transformation im 3d Raum; Wieso 4*4 und ...
	\end{subtask}
	
	\begin{subtask}[credit=2 P]
		Matrizen für Translation und Rotation um y-Achse aufstellen
	\end{subtask}
		
	\begin{subtask}[credit=3 P]
		Punkt (1,7,4) translieren um $d_x=1$, $d_y=3$ und $d_z=4$
	\end{subtask}
	
	\begin{subtask}[credit=2 P]
		Ergebnis von voriger Aufgabe 30° um die z-Achse drehen
	\end{subtask}
\end{task}

\begin{task}[credit=15{,}5 P]{Datenmanagementsysteme}
	% Reihenfolge?
	\begin{subtask}[credit=5 P]
		 Produktrepräsentationen von Einzeltteilmodellierung, CFD, FEM, RPT CAD/CAM
	\end{subtask}
	
	\begin{subtask}[credit=2 P]
		
	\end{subtask}
	
	\begin{subtask}[credit=2{,}5 P]
		Was sind Inputdeck und Outputdeck und welche "Informationen" oder so beeinhalten sie
	\end{subtask}
	
	% Aus Skript so ungefähr zur Einleitung:
	Unterschiedliche Gründe für den Einsatz von Produktdatenmanagementsystemen sowie deren konkrete Einsatzziele ergeben sich aus den Zielgrößen der Produktentwicklung:
	\begin{itemize}
		\item Erhöhung der Produktqualität
		...
	\end{itemize} %?
	
	\begin{subtask}[credit=4 P]
		Funktionen von Produktdatenmanagementsystemen
		+ Methoden der Verbesserung
	\end{subtask}
	
	\begin{subtask}[credit=2 P]
		2 Beispiele der Metadaten in SDM
	\end{subtask}
\end{task}

\begin{task}[credit=18{,}5 P]{CAE-Prozessketten}
	% Reihenfolge?
	\begin{subtask}[credit=2 P]
		Ordnen Sie die folgenden Lastfälle den "Prozessen" zu:
		MKS, FEM x2, CFD, Kräfte/Momente, Druck, Bewegungsbahn, Temperatur
		Strömungsimulation, Bewegungsabläufe, Temperaturverteilung/ Wärmeübertragung, Strukturmechanik %ähnlicher Wortlaut
	\end{subtask}
	
	\begin{subtask}[credit=4 P]
		Unterscheidungskriterium für MKS Simulationen und in welche 2 Arten die Mehrkörper analyse unterteilt werden
		+ 3 Anwendungsfälle %?
	\end{subtask}	
	
	\begin{subtask}[credit=4 P]
		FEM-Prozesskette als Stichpunkte
		% welche verfahren es zur netzerstellung von fe gibt
	\end{subtask}
	
	\begin{subtask}[credit=2{,}5 P]
		Welche Randbedingungen können in der CFD angegeben werden?
	\end{subtask}
	
	\begin{subtask}[credit=4 P]
		
	\end{subtask}
	
	\begin{subtask}[credit=2 P]
		Nennen sie die beiden Formulierungen in der CFD (Netzgenerierung)
	\end{subtask}
\end{task}

\begin{task}[credit=23 P]{FEM}
	\begin{subtask}[credit=5 P]
		Nennen Sie die Berechnungsschritte, die zur Berechnung der an Knoten wirkenden Kräfte mittels der FE-Methode durchgeführt werden müssen.
		Beachten Sie dabei auch die Reihenfolge.
	\end{subtask}
	
	\begin{subtask}[credit=2 P]
		Geben Sie die Federkonstante $k_i=\frac{EA}{L_i}$ der drei Stäbe in Abhängigkeit der gegebenen Größen E, A und L an.
		Beachten Sie $L_1 = L, L_1 \ne L_2 = L_3$.
	\end{subtask}
	
	Zur Ermittlung der wirkenden Kräfte muss das Gleichungssystem $f = K * u$ gelöst werden.
	
	\begin{subtask}[credit=16 P]
		Ermitteln Sie die globale Steifigkeitsmatrix $\mathbf{K}$ für das gegebene Stabwerk.
		Dokumentieren Sie Zwischenergebnisse.
		Beachten Sie das gegebene Koordinatensystem.
		Nehmen Sie nur für diesen Aufgabenteil für die Federkonstante an: $k_1 = k_2 = k_3 = k$.
		\textit{Hinweis: Das Lösen der Meister Zahlenwerte run}
	\end{subtask}
\end{task}

\end{document}
